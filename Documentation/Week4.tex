\documentclass{article}
\usepackage{amsmath}
\usepackage{caption}
\usepackage{graphicx}
\usepackage{yfonts}
\usepackage{listings}
\usepackage{lipsum}
\usepackage{multicol}
\usepackage{hyperref}
\usepackage{color}
	\title{PHYS488: Week3 - Objective-orientation introduction}
	\author{Zachary Humphreys \\ 200951438}	
	\lstdefinestyle{custom}{basicstyle=\tiny,language=java,showspaces=false,showstringspaces=ffalse,tabsize=1,keepspaces=true}
	\lstset{style=custom}
\begin{document}
	\maketitle
	\begin{abstract}
		\begin{center}
		\textit{}
		\end{center}
	\end{abstract}
%Main Body of text in 2Column format
\begin{multicols}{2}
\section{Task 1}
This first section required the near-Gaussian graph generated from last week to be re-imported into the code for this week so it could be compared with the expected values, and a second method for creating a Gaussian histogram which used the \textbf{java.util.Random} class to generate random Gaussian values. This was done by creating another Histogram instance with identical parameters on line 24 and adding into the for-loop used for filling the histograms a separate section for the second histogram which had to scale the randomly generated Gaussian value by $0.5$ as the default values produced have a standard deviation of 1 and was done on lines 37 and 38 and then printed to console and to a .csv file with lines 44 and 45.
\begin{table*}[t]
	\hrule
	\begin{lstlisting}
24 Histogram nextGaussHist = new Histogram(50, -2, 2); 
36 //For Next Gauss                             
37 double value_next = randGen.nextGaussian()*0.5;
38 nextGaussHist.fill(value_next);
44 nextGaussHist.print();                            
45 nextGaussHist.writeToDisk("nextgauss_test.csv");
	\end{lstlisting}
	\hrule
\end{table*}
The output data was then put into the same spreadsheet, and had their graphs overlayed upon the same axis together with the expected 
\end{multicols}
\end{document}